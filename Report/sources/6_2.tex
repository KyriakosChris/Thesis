Despite the benefits of our approach, there are some drawbacks to consider. First of all, there are some harsh requirements in the input video that we are asking from the user, that reduces the number of motions that someone can produce. \\

Furthermore, the Skeleton that we produce has only 17 bones, containing only the most essential human bones. In contrast, other methods that produce computerized motion capture clips, (with mocap-suits) can produce motion data that contains over 100 bones, with great accuracy. Therefore, this number is very low, but if we increase it with our system and the current models that we are using the accuracy as well as the quality of the estimated motion data will drop significantly.\\

Moreover, our approach requires the user to have an above-average system in order to run our methods. Again, someone with a mocap-suit can produce their motion data clip in every system that can handle at least 3D computer graphics software like Blender.\\

Finally, we used the best pre-trained models that we could find online. The researchers that created these models, will continue to improve them, however, in our program, the models would stay stable. More specifically, we will have to keep updating our models with the newest versions of them in order to have a state-of-the-art method.\\