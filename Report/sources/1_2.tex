A new approach to resolving these issues is to address the problem with Neural Networks. More specifically, in this thesis, we propose a composition of a deep neural network \cite{Exploiting temporal information for 3D pose estimation} that will estimate the 3D human pose, and the output of the DNN will be saved into a Bio-vision Hierarchy (BVH) file. The task is to predict a pose skeleton for the person in each image of a video. Currently, human pose estimation is one of the challenging fields of study in computer vision which aims in determining the position or spatial location of body keypoints (parts/joints) of a person from a given image or video. 

The first step is to use a pre-trained deep neural network that can estimate these 2D keypoints. Afterward, \cite{3D Human Pose Estimation from Deep Multi-View 2D Pose,3D Human Pose Estimation Using Convolutional Neural Networks with 2D Pose Information} in order to achieve our goal we will use another pretrain model that will be fed with a 2D pose estimation as the input, estimates the 3D pose. However, estimating a 3D human pose from a single image is more challenging than 2D cases due to the lack of depth information, so the accuracy drops significantly. In order to increase the quality of the estimation, we suggest using the butter-worth filter, which removes noise from data without affecting the motion. After obtaining the smoothed 3D human pose, we need to create a skeleton, based on the joints that the first model estimated and import the keypoints to each corresponding bone. Finally, the BVH file will be created and it can be further cleaned manually by an animator in any 3D computer graphics software (Blender, Maya, AutoDesk). 

However, the code implementation will be in python, and most animators may struggle to install and run the code. In order to avoid this issue, we developed a Windows Application, that anyone can use that runs the python code of the Thesis. 