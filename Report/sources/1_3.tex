This thesis is organized in 6 Chapters:

	\begin{itemize}
		\item Chapter 1\\\\
            In this Chapter we discuss about the problems of the MoCap clips production and propose a solution to it. 
		\item Chapter 2\\\\
            In this Chapter, we provide the reader with a overview of some knowledge about MoCap Clips, as well as some pioneer studies and state the novelties of this study. Then, we present an approach with Generative Adversarial Networks (GAN's) that was a possible solution of the problematic situation, and the reasons that we aborted this solution.
		\item Chapter 3\\\\
		    In this Chapter, we show the reader the minimum requirements of the hardware that the proposed DNN needs in order to run. In addition, we suggest to the users the best case scenarios of the input, in order to improve the algorithm estimation. Finally, we present the Interface of the Windows Application UI.  
		\item Chapter 4\\\\
            In this Chapter, we explain the implementation of the algorithm as well as some important knowledge that is covered in the thesis. More specifically, the reader will be introduced to some pretrained models that are essential to the solution, and the procedure that these models use in order to estimate the orientation and location of the person. Having created the BVH file, we develop an Animator Tool, that animators can use to improve the results. Finally, we present the python library that allows us to convert the python code to a Windows Application, as well as the features of the Application. 
		\item Chapter 5\\\\
            In this Chapter, we compare the performance of the pretrained models in the ideal circumstances for each model. Moreover, we calculate the accuracy between a real and generated clip, and the reader will learn the circumstances that our model needs in order to work at its best.
		\item Chapter 6\\\\
            In this Chapter, we summarize our results and discuss our contribution. Finally, we address the disadvantages of our approach as well as some future ideas that would improve our work. 

	\end{itemize}