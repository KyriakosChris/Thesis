Unfortunately, the results of this approach were terrible. The main reason was the small dataset that we were able to find for free. Even though we tried to do data augmentation in our dataset, there is a limitation. The main limitation of data augmentation arises from the data bias, i.e. the augmented data distribution can be quite different from the original one. This data bias leads to a sub-optimal performance of existing data augmentation methods. In particular, the generator could not improve his results, so the output was something a little better than the  noise z using a normal or uniform distribution that we sampled in the generator in the initial phase and the humanoid was doing independent random moves in each frame of the motion. The dataset limitation in this area is something that many researchers have noticed, and the goal was to try to increase the dataset even with poorer quality with this model. 