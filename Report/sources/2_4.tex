All these methods try to estimate the 3D human Pose from a video. Another very interesting approach is to try to generate a BVH file that contains the mocap human pose information. The neural network that can achieve such task is a Generative adversarial network (GAN) \cite{ANIMGAN}. A GAN contains two neural networks the discriminator and the generator. Before the training starts, the former learns to separate fake data from real, and in the case of a fake, it includes how far it is from being real. The latter is creating some data, and its goal is to fool the discriminator by creating virtual data that looks real. Moreover in GAN's many researchers uses a type of layer, Long Short-Term Memory (LSTM) that are a type of recurrent neural network capable of learning order dependence in sequence prediction problems. Simply, this layer tries to find every association of the data, in our case, the association between the neighbours frames of the video. So, this paper, proposed a spatiotemporally-conditioned
GAN that generates a sequence that is similar to a given sequence in terms of semantics and spatiotemporal dynamics.Using LSTM-based generator and graph ConvNet discriminator, this system is trained end-to-end on a large gathered dataset of gestures, expressions, and actions.