Converting a raw Python code into a Windows Application, can not be done in many different ways. The way that we chose to follow is to use the Tkinter Python library. Tkinter is the standard GUI library for Python. Python when combined with Tkinter provides a fast and easy way to create GUI applications. This Python framework provides an interface to the Tk toolkit and works as a thin object-oriented layer on top of Tk. The Tk toolkit is a cross-platform collection of  'graphical control elements', aka widgets, for building application interfaces.\\

Many people may think that this library will reduce the raw python performance. At first glance, it is reasonable to assume that Tkinter is not going to perform well. While this opinion is true, in modern systems it really does not matter too much.\\

Tkinter (Tk) consists of a number of modules \cite{Tkinter 1,Tkinter 2}. The Tk interface is located in a binary module named \_tkinter (this was Tkinter in earlier versions). This module contains the low-level interface to Tk, and should never be used directly by application programmers. It is usually a shared library (or DLL), but might in some cases be statically linked with the Python interpreter. In addition to the Tk interface module, Tkinter includes a number of Python modules. The two most important modules are the Tkinter module itself, and a module called Tkconstants.\\

The main goal is to create a .exe file that the user will open in order to run this Application. There is a library in python which is called pyinstaller that can import all the necessary libraries that the raw python code uses into that .exe file. By doing so, the user can run this application without having installed Python on his system. However, in order to use this library, (pyinstaller) there are some requirements. This library does not yet recognize all the python libraries, and we use some less-known. Therefore, we can not import these libraries into the .exe file and the code crashes. Again there is another way to convert a python code into a .exe file. We wrote a .bat file that finds the path of our main.py file and runs the code. Then, we convert this .bat file with another windows Application tool, the Advanced BAT to EXE Converter v4.23. So, the user can run the code from a .exe file, but he must install all the necessary libraries that we are using into his system. In python, this procedure is quite easy. We will upload a requirements.txt where we will write all the libraries with their version. In this case, the user will have to right only one command (pip install requirements.txt) and he will be ready to run our code. This also allows the experience with python and machine learning users to modify our code to their preferences. 