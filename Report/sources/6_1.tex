As we discussed in the previous Chapters, we created a Windows Application that anyone can convert a human motion from a video into a computerized motion, which he will be able to import into Game Engines like Unity.\\ 

At the moment the most popular and accurate method to do such a thing is to use a motion-capture suit. However these suits cost over 3000 dollars, so we proposed a method that only needs an input video with a resolution of at least 480p and an above-average system that nowadays the majority of people, especially the thesis targeted audience already have. Therefore, we give the option to studios, or animators that does not have the budget to buy a motion capture suit, to be able to create their own computerized motion capture clips. Without our method as well as a motion capture suit, someone in order to create a computerized mocap clip, use a technique similar to stop motion, using a 3D computer graphics software. Obviously, this procedure is too time-consuming, so we can reduce a lot the working time of these people. Finally, in order to reduce even more their working time, we created some filtering and editing tools that will automatically, improve the quality of a produced computerized motion capture.