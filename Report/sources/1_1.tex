Motion Capture (MoCap) is a cutting-edge method of capturing all or part of an actor's performance so that it can be translated into the action of a computer generated 3D character on screen. More specifically, the exact movements of the actor are captured and rendered onto the digital character. 

In the last years, there is a growing demand for MoCap in many fields including interactive virtual reality, film production, animation and so forth \cite{Efficient Content-Based Retrieval of Motion Capture Data}. However, capturing motions when needed is often not practical as motion capture systems are expensive and the capture processes are complex in general. It is often desirable to retrieve and reuse motion clips that have been captured before and stored in databases. Straightforwardly, the retrieval may be done based on text labels of motion clips. 

Therefore, using MoCap clips for production consists of the following problems. Firstly, there are a limited amount of MoCap clips that can be used and using something more unique means that the production must have the budget and the time to support it. Secondly, the majority of the stored databases of MoCap are not free to use or have poor quality.
